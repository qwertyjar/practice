\documentclass{article}
\usepackage{graphicx} % Required for inserting images
\usepackage[utf8]{inputenc}
\usepackage[english,russian]{babel}
\usepackage{indentfirst}
\usepackage{misccorr}
\usepackage{amsmath}
\usepackage{float}
\begin{document}
  \begin{titlepage}
    \begin{center}
      \large
      МИНИСТЕРСТВО НАУКИ И ВЫСШЕГО ОБРАЗОВАНИЯ РОССИЙСКОЙ ФЕДЕРАЦИИ
      
      Федеральное государственное бюджетное образовательное учреждение высшего образования
      
      \textbf{АДЫГЕЙСКИЙ ГОСУДАРСТВЕННЫЙ УНИВЕРСИТЕТ}
      \vspace{0.25cm}
      
      Инженерно-физический факультет
      
      Кафедра автоматизированных систем обработки информации и управления
      \vfill

      \vfill
      
      \textsc{Отчет по практике}\\[5mm]
      
      {\LARGE Программаная реализация численного метода \textit{Текст из задания по варианту.}}
      \bigskip
      
      1 курс, группа 1ИВТ АСОИУ
    \end{center}
    \vfill
    
    \newlength{\ML}
    \settowidth{\ML}{«\underline{\hspace{0.7cm}}» \underline{\hspace{2cm}}}
    \hfill\begin{minipage}{0.5\textwidth}
      Выполнил:\\
      \underline{\hspace{\ML}}Я.\,И.~Бармин\\
      «06» \,06. 2024 г.
    \end{minipage}%
    \bigskip
    
    \hfill\begin{minipage}{0.5\textwidth}
      Руководитель:\\
      \underline{\hspace{\ML}} С.\,В.~Теплоухов\\
      «06» \,06. 2024 г.
    \end{minipage}%
    \vfill
    
    \begin{center}
      Майкоп, 2024 г.
    \end{center}
  \end{titlepage}
 \section{Введение}
\label{sec:intro}
% Что должно быть во введении
\begin{enumerate}
 \item Вариант 6 - Найти ранг матрицы.
 \item Пример кода, решающего данную задачу
 \item Метод нахождения ранга матрицы
 \item Скриншот программы
\end{enumerate}


\section{Ход работы}
\label{sec:exp}

\subsection{Код приложения}
\label{sec:exp:code}
\begin{verbatim}
import numpy as np

def input_matrix():
    while True:
        try:
            rows = int(input("Введите количество строк: "))
            cols = int(input("Введите количество столбцов: "))
            if rows <= 0 or cols <= 0:
                raise ValueError("Количество строк и столбцов
                должно быть положительным числом.")
            matrix = []
            print("Введите элементы матрицы:")
            for i in range(rows):
                row = list(map(float, input().split()))
                if len(row) != cols:
                    raise ValueError("Количество элементов в строке
                    должно быть равно количеству столбцов.")
                matrix.append(row)
            return np.array(matrix)
        except ValueError as e:
            print("Ошибка:", e)

def calculate_rank(matrix):
    return np.linalg.matrix_rank(matrix)

def print_matrix_rank(matrix):
    rank = calculate_rank(matrix)
    print("Ранг матрицы: ", rank)

def main():
    while True:
        print("\nМеню:")
        print("1. Ввести матрицу")
        print("2. Показать введенную матрицу")
        print("3. Показать ранг матрицы")
        print("4. Выход")

        choice = input("Выберите действие (1/2/3/4): ")

        if choice == '1':
            matrix = input_matrix()
            print("Матрица успешно введена.")
        elif choice == '2':
            if 'matrix' in locals():
                print("Введенная матрица:")
                print(matrix)
            else:
                print("Сначала введите матрицу (выберите 1 из меню)")
        elif choice == '3':
            if 'matrix' in locals():
                print_matrix_rank(matrix)
            else:
                print("Сначала введите матрицу (выберите 1 из меню)")
        elif choice == '4':
            print("Выход из программы.")
            break
        else:
            print("Неверный выбор. Попробуйте еще раз.")

if __name__ == "__main__":
    main()

\end{verbatim}

\subsection{Метод нахождения ранга матрицы}
\label{sec:mathexample}

Нахождение ранга матрицы
\([[ 0.  3.  2.  1.]
 [-1.  1.  4.  1.]
 [ 3.  3. 12.  2.]]\):
\begin{verbatim}
Для нахождения ранга матрицы следуй этим шагам:
1. Привести матрицу к ступенчатому виду, используя элементарные преобразования строк
(например, приведение строк к нулевым элементам под главной диагональю).
2. Посчитать количество ненулевых строк в полученной ступенчатой матрице.
Это и будет ранг матрицы.
\end{verbatim}

\section{Изображение с примером нахождения ранга матрицы и результат выполнения программы}
\label{sec:picexample}
\begin{figure}[h]
	\centering
	\includegraphics[width=0.7\textwidth]{rang.jpg}
	\caption{Ранг матрицы}\label{fig:par}
\end{figure}
Пример нахождения ранга матрицы представлен на рис.~\ref{fig:par}.

Результат выполнения программы представлен на рис.~\ref{fig:par2}.
\label{sec:picexample}
\begin{figure}[H]
	\centering
	\includegraphics[width=0.4\textwidth]{result.jpg}
	\caption{Результат выполнения программы}\label{fig:par2}
\end{figure}



\section{Пример библиографических ссылок}

Для изучения «внутренностей» \TeX{} необходимо 
изучить~\cite{Knuth-2003}, а для использования \LaTeX{} лучше
почитать~\cite{Lvovsky-2003, Voroncov-2005}.

\begin{thebibliography}{9}
\bibitem{Knuth-2003}Кнут Д.Э. Всё про \TeX. \newblock --- Москва: Изд. Вильямс, 2003 г. 550~с.
\bibitem{Lvovsky-2003}Львовский С.М. Набор и верстка в системе \LaTeX{}. \newblock --- 3-е издание, исправленное и дополненное, 2003 г.
\bibitem{Voroncov-2005}Воронцов К.В. \LaTeX{} в примерах. 2005 г.
\end{thebibliography}

\end{document}
